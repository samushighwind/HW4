
\section{Conclusions}
\label{sec:concl}

This study sought to establish a novel method for assigning courses at
Davidson College by framing the problem as a constraint satisfaction
problem. After a comparative analysis of the old system of Webtree and
our new method, using a pseudoboolean SAT solver, we conclude that our
solution assigns courses in a manner that gives more students all four
courses (i.e. a full course-load), while giving all students more
courses that they want than Webtree does. Our best end state was
obtained in the absence of the additional constraint on minimum number
of courses granted for each student. Without this state, our model
performs better than Webtree by providing a just and appealing
distribution of courses for all students. Therefore, if an overhaul to
the method of assigning courses at Davidson College is in the near
future, framing it as a constraint satisfaction problem should be a
strong consideration of the Office of the Registrar. 


Future research should establish a new method and interface for
ranking courses. It should be simplified from the current state, while
still giving students a chance to customize different contingencies
depending on whether or not they receive certain courses. Importantly, more
research should examine what sorts of additional constraints or constraint
modifications could result in improved scheduling, as well as suitable ways to
incorporate desired hierarchical schemas such as class and lottery number. Does
there exist a constraint schema that can supply a higher number of completed
schedules than vanilla Webtree, while simultaneously returning a higher number
of fulfilled requests and a higher score of course satisfaction?
