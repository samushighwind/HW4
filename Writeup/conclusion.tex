
\section{Conclusions}
\label{sec:concl}

This study sought to establish a novel method for assigning courses at
Davidson College by framing the problem as a constraint satisfaction
problem. After a comparative analysis of the old system of Webtree and
our new method, using a psuedoboolean SAT solver, we conclude that our
solution assigns courses in a manner that gives more students all four
courses (i.e. a full course-load), as well as three courses. While the
old Webtree algorithm assigns a higher average of courses to each
student (3.735) than our approach without minimum courses as a
constraint (3.716), adding a constraint that requires each student to
receive a minimum of 3 courses was found to result assigning the
maximum average number of courses to each student
(3.746). Additionally, while not adding constraints that require a
minimum number of courses to be fulfilled per student resulted in a
lower average courses per student, students can easily add more
courses after the algorithm has run during the add/drop period. 

Future research should examine...