
\section{Experiments}
\label{sec:expts}

Each variable $x_i$ represented one full Webtree request by a
student. Thus, setting that variable true indicates that that course
is granted to that particular student. Our constraint set consists of
four subsets: ensuring that each student would not receive the same
course (of any and all sections) more than once, ensuring that each
student would not receive more than four courses, ensuring that
enrollment for each course fell under or equal to the course ceiling,
and ensuring that each student received at least $n$ courses. The
last constraint is varied for $n$ from {0-4}, which allows us to treat it as
a ``soft constraint'', as setting $n=0$ is similar to saying that
the constraint does not need to be followed. 

To create our minimization function, we create a new function $rank'(i)$:

\begin{equation} rank'(i) = -8 + rank(i) \end{equation}

The function $rank'(i)$ is used so that it can be minimized for optimal
results, with -7 being the most desirable rank in this scheme and -1 being
the least desirable. Thus our minimization function minimizes the summation
in figure 2, though with $rank'(i)$ substituted in for $rank(i)$.

We solve course scheduling given a set of roughly 40,000 requests
from the Spring 2015 Webtree data set.

We use Sat4j's psuedoboolean solver, a free open-source java implementation, to solve this constraint
satisfaction problem \cite{sat4j}.
