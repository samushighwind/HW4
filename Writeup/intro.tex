
% The \section{} command formats and sets the title of this
% section. We'll deal with labels later.
\section{Introduction}
\label{sec:intro}

With over 400 courses offered every semester at Davidson College,
assigning students courses based on their preferences is not a trivial
task. For years, the Office of the Registrar has used ``Webtree'', a
system designed to be fair yet give students preferential treatment
according to seniority. Webtree, untouched in the past several decades, remains a constant
source of confusion, and later in the course selection process,
frustration. At first, students are overwhelmed by the non-intuitive
user interface. Once Webtree has run, students are then disappointed
by the courses they receive. Due to the static nature of Webtree, it
is not uncommon for students to receive only one or two courses after
an entire run of Webtree (with four courses being a full
course-load). 

Course scheduling is a common problem for artificial intelligence
researchers and much work has gone into efficient ways to assign
courses. However, Davidson College's Webtree is a unique problem that
has not been studied yet, to our knowledge. Much of the prior research
into course scheduling involves placing courses in appropriate
classrooms and at appropriate times. Davidson College's course
selection is unique because course ceilings are strictly enforced, to
ensure that class enrollment sizes are kept small to give students the
necessary attention to foster a positive learning
environment. Additionally, room selection is rarely an issue at
Davidson College because very few classes exceed an enrollment of $30$
students, and so most classes can fit into most rooms.

Recently, researchers have been framing the university course
selection problem as a constraint satisfaction problem
\cite{darden}. Constraint satisfaction problems maximize or minimize
an objective function over a set of constraints with variables in a
given domain. By setting constraints as actual limitations that exist,
such a course must obey the given course ceiling, the course
scheduling process can be made equitable and processed quickly.
 
The rest of the paper is organized beginning with a background section
explaining Webtree as well as the type of pseudoboolean constraint
problem we use to solve Webtree more efficiently. Following that, our 
experiments section details our pseudoboolean constraints and minimization
function, as well as the process by which we ranked course requests
for maximum student satisfaction. Our results section shows how request
fulfillment and satisfaction vary across the different constraint schemas
we tested, and finally, we conclude with recommendations on further research.

% Citations: As you can see above, you create a citation by using the
% \cite{} command. Inside the braces, you provide a "key" that is
% uniue to the paper/book/resource you are citing. How do you
% associate a key with a specific paper? You do so in a separate bib
% file --- for this document, the bib file is called
% project1.bib. Open that file to continue reading...

% Note that merely hitting the "return" key will not start a new line
% in LaTeX. To break a line, you need to end it with \\. To begin a 
% new paragraph, end a line with \\, leave a blank
% line, and then start the next line (like in this example).
%Overall, the aim in this section is context-setting: what is the
%big-picture surrounding the problem you are tackling here?

