
\section{Results}
\label{sec:results}

% Present the results of your experiments. Simply presenting the data is
% insufficient! You need to analyze your results. What did you discover?
% What is interesting about your results? Were the results what you
% expected? Use appropriate visualizations. Prefer graphs and charts to
% tables as they are easier to read (though tables are often more
% compact, and can be a better choice if you're squeezed for space).
% \textbf{Always} include information that conveys the uncertainty in
% your measurements: mean statistics should be plotted with error bars,
% or reported in tables with a $\pm$ range. The $95\%$-confidence
% interval is a commonly reported statistic.

The results from our experiment are presented in Table
~\ref{fig:mainresults}. We see the comparison between Webtree's
algorithm and our Constraint Satisfaction pseudoboolean algorithm, for
three different trials with a different minimum number of
courses. Webtree assigned an average course rank of -6.30, which is
worse than any of the average course ranks for our algorithm (recall
that since this is a \emph{minimization} problem, a smaller rank is
better). Our model yielded average course ranks of -6.47, -6.46. and
-6.43 for minimum course constraints of 0, 2, and 3,
respectively. This is unsurprising, since adding a constraint to give
each student a minimum number of courses likely will reduce the rank
of the courses that students are given, a sacrifice that is made in
order to ensure students receive courses.

\begin{table}[h]
\begin{tabular}{lllll}
 & \begin{tabular}[c]{@{}l@{}}Min.\\ courses\end{tabular}
 & \begin{tabular}[c]{@{}l@{}}Avg. \\ rank\end{tabular}
 & \begin{tabular}[c]{@{}l@{}}Avg.\\ courses\\ granted\end{tabular}
 & \begin{tabular}[c]{@{}l@{}}Students\\ with 4\\
     courses \end{tabular} \\ \\
\begin{tabular}[c]{@{}l@{}}Webtree\\
  \\\end{tabular}& \begin{tabular}[c]{@{}l@{}}N/A \\ \\\end{tabular}
 & \begin{tabular}[c]{@{}l@{}}-6.30\\ \\\end{tabular}
 &\begin{tabular}[c]{@{}l@{}}3.74\\ \\\end{tabular}
 & \begin{tabular}[c]{@{}l@{}}1396\\ \\\end{tabular} \\ 
\begin{tabular}[c]{@{}l@{}}CSP\\ \\\end{tabular}
 & \begin{tabular}[c]{@{}l@{}}0\\ \\\end{tabular} %
 & \begin{tabular}[c]{@{}l@{}}-6.47\\ \\\end{tabular}
 &\begin{tabular}[c]{@{}l@{}}3.72\\ \\\end{tabular}
 & \begin{tabular}[c]{@{}l@{}}1440\\ \\\end{tabular} \\ 

\begin{tabular}[c]{@{}l@{}}CSP\\ \\\end{tabular}
 & \begin{tabular}[c]{@{}l@{}}2\\ \\\end{tabular}
 & \begin{tabular}[c]{@{}l@{}}-6.46\\ \\\end{tabular}
 &\begin{tabular}[c]{@{}l@{}}3.72\\ \\\end{tabular} & \begin{tabular}[c]{@{}l@{}}1427\\ \\\end{tabular} \\ 

\begin{tabular}[c]{@{}l@{}}CSP\\ \\\end{tabular}
 & \begin{tabular}[c]{@{}l@{}}3\\ \\\end{tabular}
 & \begin{tabular}[c]{@{}l@{}}-6.43\\ \\\end{tabular}
 &\begin{tabular}[c]{@{}l@{}}3.75\\ \\\end{tabular} & \begin{tabular}[c]{@{}l@{}}1354\\ \\\end{tabular} \\ 

\end{tabular}
\caption{Our results of course assignment, compared to Webtree as a
  control. CSP stands for Constraint Satisfaction Problem, the method
used in this experiment.}
\label{fig:mainresults}
\end{table}

However, unless we add a constraint to give each student a minimum of
three courses, Webtree assigns an average of more courses. This is a
tradeoff that we are willing to accept, since the difference is
trivial (Webtree assigns an average of 3.74 courses, and ours assigns
an average of 3.72 courses), and the average rank of the courses is
much better in our model. 

Interestingly, the number of students that receive four courses
decreases as we increase the number of courses that students are
required to receive. As we increase the number of forced required
courses, we take courses away from those who had all four and give
them to those who had less than the minimum number. Even though less
students receive four courses, the average number of courses received
increases. Essentially, we are making a trade off that promotes more
equality while still ensuring that students receive courses that they
rank highly on their Webtree form. 


% \subsection{Embedding Pictures}
% \label{subsec:pics}

% See the source code (\texttt{results.tex}) for instructions on how to
% insert figures (like figure~\ref{fig:tex}) or plots into your
% document.


% \begin{figure}[htb]

%   \centering  % centers the image in the column

%   % replace the second argument below with your filename. I like to
%   % place all my figures in a sub-directory to keep things organized
%   \includegraphics[width=0.37\textwidth]{figs/file_extensions.png}

%   % *Every* figure should have a descriptive caption.
%   \caption{On the trustworthiness of \LaTeX. Image courtesy of \texttt{xkcd}.}

%   % The label is a handle you create so that you can refer to this
%   % figure (using the \ref{} command) from other parts of your
%   % document. LaTeX automatically renumbers figures and updates
%   % references when you recompile, so you should do it this way rather
%   % than hard-coding in references. Notice that I've also been
%   % creating labels for the various sections in the document; I could
%   % use \ref{} command to refer to those sections using their labels
%   % too.
%   \label{fig:tex}

% \end{figure}

