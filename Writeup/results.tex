
\section{Results}
\label{sec:results}

% Present the results of your experiments. Simply presenting the data is
% insufficient! You need to analyze your results. What did you discover?
% What is interesting about your results? Were the results what you
% expected? Use appropriate visualizations. Prefer graphs and charts to
% tables as they are easier to read (though tables are often more
% compact, and can be a better choice if you're squeezed for space).
% \textbf{Always} include information that conveys the uncertainty in
% your measurements: mean statistics should be plotted with error bars,
% or reported in tables with a $\pm$ range. The $95\%$-confidence
% interval is a commonly reported statistic.

\begin{table}[h]
\begin{tabular}{lllll}
 & \begin{tabular}[c]{@{}l@{}}Min.\\ courses\end{tabular} & \begin{tabular}[c]{@{}l@{}}Avg. \\ rank\end{tabular} & \begin{tabular}[c]{@{}l@{}}Avg.\\ courses\\ granted\end{tabular} & \begin{tabular}[c]{@{}l@{}}Students\\ with 4\\ courses\end{tabular} \\
Webtree & N/A & -6.30 & 3.74 & 1396 \\
\begin{tabular}[c]{@{}l@{}}Constraint\\ Satisfaction\\ Problem\end{tabular} & 0 & -6.47 & 3.72 & 1440 \\
\begin{tabular}[c]{@{}l@{}}Constraint\\ Satisfaction\\ Problem\end{tabular} & 2 & -6.46 & 3.72 & 1427 \\
\begin{tabular}[c]{@{}l@{}}Constraint\\ Satisfaction\\ Problem\end{tabular} & 3 & -6.43 & 3.75 & 1354
\end{tabular}
\end{table}


% \subsection{Embedding Pictures}
% \label{subsec:pics}

% See the source code (\texttt{results.tex}) for instructions on how to
% insert figures (like figure~\ref{fig:tex}) or plots into your
% document.


% \begin{figure}[htb]

%   \centering  % centers the image in the column

%   % replace the second argument below with your filename. I like to
%   % place all my figures in a sub-directory to keep things organized
%   \includegraphics[width=0.37\textwidth]{figs/file_extensions.png}

%   % *Every* figure should have a descriptive caption.
%   \caption{On the trustworthiness of \LaTeX. Image courtesy of \texttt{xkcd}.}

%   % The label is a handle you create so that you can refer to this
%   % figure (using the \ref{} command) from other parts of your
%   % document. LaTeX automatically renumbers figures and updates
%   % references when you recompile, so you should do it this way rather
%   % than hard-coding in references. Notice that I've also been
%   % creating labels for the various sections in the document; I could
%   % use \ref{} command to refer to those sections using their labels
%   % too.
%   \label{fig:tex}

% \end{figure}

